\chapter{Podsumowanie}
\thispagestyle{chapterBeginStyle}

\iffalse
W podsumowanie należy określić stan zakończonych prac projektowych i implementacyjnych. Zaznaczyć, które z zakładanych funkcjonalności systemu udało się zrealizować. Omówić aspekty pielęgnacji systemu w środowisku wdrożeniowym. Wskazać dalsze możliwe kierunki rozwoju systemu, np.\ dodawanie nowych komponentów realizujących nowe funkcje.

W podsumowaniu należy podkreślić nowatorskie rozwiązania zastosowane w projekcie i implementacji (niebanalne algorytmy, nowe technologie, itp.).
\fi

Podsumowując implementację udało się zakończyć sukcesem. Zawiera ona pełną funkcjonalność abstrakcyjnej maszyny Warrena, pozwala rozdzielenie kompilacji i wykonywania i pozwala na łatwe testowanie działania i programów. Mimo to pozostaje jeszcze wiele ścieżek rozbudowy.\\
Zaproponowana implementacja o ile działa, to działa o wiele wolniej od innych łatwo dostępnych implementacji. Dwa możliwe sposoby na przyspieszenie jej działania to: wprowadzenie nowych, bardziej wyspecjalizowanych instrukcji (np. \texttt{get\_list}) i zmiana sposobu przechowywania kodu z tekstowego na reprezentowanego przez liczby naturalne.\\
Inną możliwością ulepszenia implementacji jest dodanie nowych funkcjonalności powszechnie dostępnych w innych implementacjach, takich jak: arytmetyka, operacje na ciągach znaków i odcięcia.
