\chapter{Opis implementacji}
\thispagestyle{chapterBeginStyle}

\iffalse
\section{Opis technologii}

Należy tutaj zamieścić krótki opis (z referencjami) do technologii użytych przy implementacji systemu.

{\color{dgray}
Do implementacji systemu użyto języka JAVA w wersji \ldots, szczegółowy opis można znaleźć w \cite{Java}. Interfejs zaprojektowano w oparciu o HTML5 i CSS3 \cite{HTML-CSS}.
}

\section{Omówienie kodów źródłowych}

{\color{dgray}
Kod źródłowy~\ref{ws} przedstawia opisy poszczególnych metod interfejsu: \texttt{WSPodmiotRejestracjaIF}. Kompletne
kody źródłowe znajdują się na płycie CD dołączonej do niniejszej pracy w katalogu \texttt{Kody} (patrz Dodatek~\ref{plytaCD}).
}

\begin{small}
\begin{lstlisting}[language=Java, frame=lines, numberstyle=\tiny, stepnumber=5, caption=Interfejs usługi Web Service: \texttt{WSPodmiotRejestracjaIF}\label{ws}., firstnumber=1]
package erejestracja.podmiot;
import java.rmi.RemoteException;
// Interfejs web serwisu dotyczącego obsługi podmiotów i rejestracji.
public interface WSPodmiotRejestracjaIF extends java.rmi.Remote{
// Pokazuje informacje o danym podmiocie.
// parametr: nrPeselRegon - numer PESEL podmiotu lub numer REGON firmy.
// return: Podmiot - obiekt transportowy: informacje o danym podmiocie.
public Podmiot pokazPodmiot(long nrPeselRegon) throws RemoteException;
// Dodaje nowy podmiot.
// parametr: nowyPodmiot - obiekt transportowy: informacje o nowym podmiocie.
// return: true - jeśli podmiot dodano, false - jeśli nie dodano.
public boolean dodajPodmiot(Podmiot nowyPodmiot) throws RemoteException;
// Usuwa dany podmiot.
// parametr: nrPeselRegon - numer PESEL osoby fizycznej lub numer REGON firmy.
// return: true - jeśli podmiot usunięto, false - jeśli nie usunięto.
public boolean usunPodmiot(long nrPeselRegon) throws RemoteException;
// Modyfikuje dany podmiot.
// parametr: podmiot - obiekt transportowy: informacje o modyfikowanym podmiocie.
// return: true - jeśli podmiot zmodyfikowano, false - jeśli nie zmodyfikowano.
public boolean modyfikujPodmiot(Podmiot podmiot) throws RemoteException;
// Pokazuje zarejestrowane podmioty na dany dowód rejestracyjny.
// parametr: nrDowoduRejestracyjnego - numer dowodu rejestracyjnego.
// return: PodmiotRejestracja[] - tablica obiektów transportowych: informacje o
// wszystkich zarejestrowanych podmiotach.
public PodmiotRejestracja[] pokazZarejestrowanePodmioty(
String nrDowoduRejestracyjnego) throws RemoteException;
// Nowa rejestracja podmiotu na dany dowód rejestracyjny.
// parametr: nrDowoduRejestracyjnego - numer dowodu rejestracyjnego.
// parametr: nrPeselRegon - numer PESEL podmiotu lub numer REGON firmy.
// parametr: czyWlasciciel - czy dany podmiot jest właścicielem pojazdu.
// return: true - jeśli zarejestrowano podmiot, false - jeśli nie zarejestrowano.
public boolean zarejestrujNowyPodmiot(String nrDowoduRejestracyjnego,
long nrPeselRegon, boolean czyWlasciciel) throws RemoteException;
// Usuwa wiązanie pomiędzy danym podmiotem, a dowodem rejestracyjnym.
// parametr: nrDowoduRejestracyjnego - numer dowodu rejestracyjnego.
// parametr: nrPeselRegon - numer PESEL podmiotu lub numer REGON firmy.
// return: true - jeśli podmiot wyrejestrowano, false - jeśli nie wyrejestrowano.
public boolean wyrejestrujPodmiot(String nrDowoduRejestracyjnego,
long nrPeselRegon) throws RemoteException;
\end{lstlisting} 
\end{small}

{\color{dgray}
Kod źródłowy~\ref{req} przedstawia procedurę przetwarzającą żądanie. Hasz utrwalany \verb|%granulacja| wykorzystywany jest do komunikacji międzyprocesowej.
}

\begin{small}
\begin{lstlisting}[language=perl, frame=lines, caption=Przetwarzanie żądania - procedura \texttt{process\_req()}\label{req}., firstnumber=86]
sub process_req(){	
  my($r) = @_;
  $wyn = "";
  if ($r=~/get/i) {
	@reqest = split(" ",$r);
	$zad = $reqest[0];
	$ts1 = $reqest[1];
	$ts2 = $reqest[2];
	@date1 = split(/\D/,$ts1);
	@date2 = split(/\D/,$ts2);
	print "odebralem: $r"; 
	$wyn = $wyn."zadanie: $zad\n";
	$wyn = $wyn."czas_od: "."$date1[0]"."-"."$date1[1]"."-"."$date1[2]"."_"."$date1[3]".":"."$date1[4]".":"."$date1[5]"."\n";
	$wyn = $wyn."czas_do: "."$date2[0]"."-"."$date2[1]"."-"."$date2[2]"."_"."$date2[3]".":"."$date2[4]".":"."$date2[5]"."\n";		
	$wyn = $wyn.&sym_sens($ts1,$ts2);
	return $wyn;
  }
  if ($r=~/set gt/i) {
	@reqest = split(" ",$r);
	$zad = $reqest[0];
	$ts1 = $reqest[1];
	$ts2 = $reqest[2];
	$gt = $reqest[2];
	dbmopen(%granulacja,"granulacja_baza",0644);
	$granulacja{"gt"}=$gt;
	dbmclose(%granulacja);
	$wyn = "\'GT\' zmienione na: $gt";
  }		
}	
\end{lstlisting} 
\end{small}


Główna pętla progarmu polega na wywołaniu instrukcji \texttt{CODE[P]} i zwiększeniu \texttt{P} o 1 do momentu dojścia do końca kodu lub w trakcie wykonywania instrkcji zwrócenia $fail$. W pierwszym wypadku pojawia wykonywanie zapytania kończy się powodzeniem i użytkownik zostaje zapytany czy chce kontynuować. Jeśli wybierze "tak" to program wraca do wykonywania zapytania, tak jakby instrukcja zwróciła $fail$. Jeśli wykowana instrukcja zwróci $fail$, to jeśli jest aktywny punkt wyboru to wyknonywana jest operacja nawrotu i zapytanie wykonywane jest dalej. Jeżeli nie ma aktywnego punktu wyboru, to wykonywanie zapytania kończy się z wynikiem $fałsz$.\\
Poniżej opisane są wszystkie obsługiwane instrukcje.

\section{\texttt{put\_structure} $f/n$,\texttt{X}$i$}

Dodaje na koniec sterty komórkę STR i komórkę funktora $f/n$, której adres zawiera. Następnie kopiuje tą komórkę STR do rejestru o adresie \texttt{X}$i$.

\section{\texttt{set\_variable} \texttt{X}$i$}

Dodaje na koniec sterty nieprzypisaną komórkę REF, a następnie umieszcza w rejestrze \texttt{X}$i$ komórkę REF z jej adresem.

\section{\texttt{set\_value} \texttt{X}$i$}

Dodaje na koniec sterty komórkę będącą kopią komórki z rejestru \texttt{X}$i$.

\section{\texttt{get\_structure} $f/n$,\texttt{X}$i$}

Wywołuje $deref(\texttt{X}i)$ i jeżeli otrzymany adres pokazuje na nieprzypisaną komórkę, to dodaje na koniec sterty komórkę STR i komórkę funktora $f/n$, której adres zawiera. Następnie wywołuje $bind(deref(\texttt{X}i),\texttt{H})$ i ustawia tryb maszyny na $write$ i adres \texttt{S} na koniec sterty. W przeciwnym wypadku jeśli $deref(\texttt{X}i)$ pokazuje na komórkę STR z adresem $addr$, to ustawia \texttt{S} na $addr+1$ i tryb maszyny na $read$.

\section{\texttt{unify\_variable} \texttt{X}$i$}

Jeśli maszyna jest w trybie $read$ to kopiuje komórkę z adresu \texttt{S} do rejestru \texttt{X}$i$.\\
Jeśli maszyna jest w trybie $write$ to dodaje na koniec stery nieprzypisaną komórkę i kopiuje ją do rejestru \texttt{X}$i$.

\section{\texttt{unify\_value} \texttt{X}$i$}

Jeśli maszyna jest w trybie $read$ to wywołuje $unify(\texttt{X}i, \texttt{S})$.\\
Jeśli maszyna jest w trybie $write$ to kopiuje komórkę z rejestru \texttt{X}$i$ na koniec sterty.

\section{\texttt{put\_variable} \texttt{X}$i$ \texttt{A}$j$}

Dodaje na koniec sterty nieprzypisaną komórkę, a następnie kopiuje ją do rejestrów \texttt{X}$i$ i \texttt{A}$j$.

\section{\texttt{put\_value} \texttt{X}$i$ \texttt{A}$j$}

Kopiuje komórkę z rejestru \texttt{X}$i$ do rejestru \texttt{A}$j$.

\section{\texttt{get\_variable} \texttt{X}$i$ \texttt{A}$j$}

Kopiuje komórkę z rejestru \texttt{A}$j$ do rejestru \texttt{X}$i$.

\section{\texttt{get\_value} \texttt{X}$i$ \texttt{A}$j$}

Wywołuje $unify(\texttt{X}i, \texttt{A}j)$.

\section{\texttt{call} $label$}

$label$ jest stringiem. Jeśli $label$ jest postaci $*/n$, gdzie $n$ jest liczbą naturalną, to ustawia \texttt{num\_of\_args}$ = n$. Wyszukuje w kodzie wiersza $p$ oznaczonego etykietą $label$. Jeśli takiej etykiety w kodzie nie ma to zwraca \texttt{fail}.
W przeciwnym wypdaku ustawia \texttt{CP}$ = $\texttt{P} i \texttt{P}$ = p - 1$.

\section{\texttt{proceed}}

Ustawia \texttt{P}$ = $\texttt{CP}.

\section{\texttt{allocate} $N$}

Tworzy nowe środowisko $env$ przechowujące obecne \texttt{E} i \texttt{CP} i wstawia je do stosu \texttt{and\_stack} na indeksie $max(B,E)+1$. Następnie przełącza się na środowisko $env$.\\
$N$ jest argumentem oznaczającym ilość rejestrów argumentów, ale ponieważ rejestry są przechowywane w tablicy dynamicznej, w obecnej wersji ten argument nic nie robi.

\section{\texttt{deallocate}}

Ustawia \texttt{P}$ = $\texttt{and\_stack[E].CP}, a następnie przestawia aktywne środowisko na środowisko o indeksie przechowywanym przez obecnie aktywne środowisko.

\section{\texttt{try\_me\_else} $label$}

Tworzy nowy punkt wyboru gdzie \texttt{BP}$ = label$. Następnie ustawia \texttt{B}$ = $\texttt{E} i \texttt{HB}$ = $\texttt{H}.

\section{\texttt{retry\_me\_else} $label$}

Przywraca stan maszyny do stanu bezpośrednio po utworzeniu ostatniego punktu wyboru. Także zamienia etykietę pamiętaną przez aktywny punkt wyboru na $label$. Nie zmienia rejestru \texttt{P}.

\section{\texttt{trust\_me}}

Przywraca stan maszyny do stanu z przed utworzenia ostatniego punktu wyboru, usuwając go. Nie zmienia rejestru \texttt{P}.

\section{Instrukcje specjalne}

Niektóre instrukcje pojawiają się w kodzie generowanym przez samą maszynę przed załadowaniem programu, ale nie powinny być zawierane przez ładowany program.

\subsection{\texttt{write}}

Wypisuje na standardowe wyjście tektową reprezentacje termu na który pokazuje adres $deref($\texttt{A0}$)$. Jeśli otrzymany adres pokazuje na nieprzypisaną komórkę, wyświetlany jest jako \texttt{\_}$mn$, gdzie $m$ oznacza block pamięci w której zanjduje się zmienna: $H$ - sterta, $X$ - rejestry tymczasowe, $A$ - rejestry argumentów, $Y$ - rejestry trwałe. $n$ oznacza indeks w tym bloku pamięci. Jeżeli otrzymany adres pokazuje na komórkę STR, której adres pokazuje na komórkę funtora $f/n$, gdzie $n > 0$ to rekurencyjnie wypisywane są też podtermy.
\fi

Do kompilacji zaproponowanej implementacji zostały użyte programy: g++ 7.5.0, flex 2.6.4, GNU Bison 3.0.4 i GNU Make 4.1. WAM został napisany w C++ w standardzie C++14 na system Linux (testowane na Ubuntu WSL).\\

\section{Implementacja WAM}

Implementacja została napisana tak, żeby zawierać pełną funkcjonalność abstrakcyjnej maszyny Warrena dla czystego Prologa i tą opisaną w poprzednim rozdziale. Mimo tego pojawiają się małe różnice w reprezentacji pamięci maszyny.\\

Pamięć maszyny zamiast być w jednym bloku w którym zawarte są wszystkie strefy\cite{instructions}, jest podzielona na bloki z których każdy odpowiada jednej strefie. Zaletą tego jest to, że aplikacji nie skończy się miejsce (zakładając że system operacyjny przydzieli jej wystarczająco miejsce), ponieważ używane struktury danych są dynamiczne i nie zaczną się na wzajem nadpisywać, jak to może się stać kiedy jest tylko jeden blok. Zmiany są też w komórkach pamięci. Kiedy istnieje więcej niż jeden blok pamięci adres nie może być już liczbą naturalną i musi oprócz tego zawierać wskaźnik na blok na który pokazuje. Do tego celu zaimplementowana jest klasa \texttt{Address}.\\
Na skutek tego klasa reprezentująca komórkę danych \texttt{DataCell} ma dwa pola: \texttt{std::string tag} i {Address addr}.\\
 Używane adresy prawie zawsze pokazują na stertę, ale umożliwianie im pokazywania na inne bloki pozwala uniknąć skomplikowanych błędów. Niektóre rejestry wciąż mogą pokazywać na tylko jeden blok i nie zawierają taga, więc zamiast komórki danych mogą reprzezentowane przez liczbę naturalną (np. \texttt{H}, \texttt{S}, \texttt{P}). Jako że wiele elementów pamięci WAM używa tylko jednego pola z komórki (\texttt{tag} lub \texttt{addr}) dla znaczącej części z nich można zmienić zupełnie ich typ tak, żeby lepiej odpowiadał ich funkcji. Bloki pamięci reprezentowane są przez różne struktury danych:\\
Strefa kodu to \texttt{std::vector<std::vector<std::string>>}, gdzie każdy element zewnętrzego wektora zawiera w kolejności nazwę instrukcji i jej argumenty, wszystkie zapisane jako ciągi znaków. W trakcie ładowania kodu wszystkie etykiety są z niego usuwane i są zamiast tego pamiętane mapie gdzie kluczem jest nazwa etykiety, a wartością numer linii, której odpowiada, przyspiesza to wywołania predykatów.\\
Dla sterty została zaimplementowana klasa \texttt{MemoryBloc} która działa jak wektor komórek pamięci, z tą różnicą, że jego operatory dostępu są zmienione i w przypadku próby dostępu do elementu z poza granic wektora, automatycznie się on rozszerza, na nowych miejscach dodając nieprzypisane komórki (komórki z tagiem " REF " i swoim własnym adresem). Symuluje to zachowanie pamięci w oryginalnym projekcie Warrena, gdzie cała pamięć od początku wypełniona jest nieprzypisanymi komórkami.\\
Główny stos (\texttt{STACK}) jest bardziej zmodyfikowany. W przypadku użycia pojedynczego stosu, przechowuje on zarówno środowiska jak i punkty wyboru\cite{WAM}. Zaletą tego rozwiązania jest to, że każdy punkt wyboru chroni środowiska pod nim na stosie przed nadpisaniem, ponieważ w przypadku nawrotu może isnieć konieczność powrotu do zdealokowanego środowiska. W zaproponawanej implementacji zamiast stosu \texttt{STACK} istnieją dwa stosy: \texttt{AND\_STACK} przechowujący środowiska i \texttt{OR\_STACK} przechowujący punkty wyboru. W ten sposób każdy z tych stosów jest zaimplementowany jako wektor jednego typu (środowiska i punkty wyboru są zaimplementowane jako swoje własne klasy z różnymi polami i bez metod). W tym przypadku zmienia się funkcja rejestru \texttt{B} (typu \texttt{int}), który dotąd pokazywał na ostatni punkt wyboru, ale w proponowanej implementacji ostatni punkt wyboru jest zawsze ostatnim elementem \texttt{OR\_STACK} i nie wymaga pamiętania jego lokalizacji. Rejestr \texttt{B} przejmuje zadanie chronienia zdealokowanych środowisk, pokazując wielkość stosu \texttt{AND\_STACK} w momencie utworzenia ostatniego punktu wyboru i zabezpieczając przed nadpisaniem środowiska starszego od tego punktu wyboru.\\
Ślad jest typu \texttt{std::vector<Address>} żeby pamiętać adresy przypisywanych zmiennych.\\
Stos \texttt{PDL} jest typu \texttt{std::stack<Address>} i jest zmienną lokalną wewnątrz funkcji \texttt{unify}, ponieważ jest wykorzystywany tylko tam i powinnien być zerowany po każdym wywołaniu unifikacji.\\
Rejestry tymczasowe, argumentów i trwałe są zaimplementowane za pomocą tego samego typu co sterta, czyli \texttt{MemoryBloc}, ponieważ tych rejestrów może być potrzebna niewiadoma ilość i nie można ich reprezentować za pomocą statycznej struktury danych.\\

W podstawowym przypadku użycia (kompilator + WAM) aplikacja najpierw ładuje do strefy kodu wbudowane predykaty, następnie kompiluje i ładuje program (jeżeli podany). W tym momencie obecna długość kodu zostaje zapamiętana. Następnie aplikacja wyświetla użytkownikowi polecenie wpisania zapytania. Kiedy użytkownik je wpisze, zapytanie jest kompilowane i dodawane do strefy kodu. Wykonywanie zaczyna się od początku instrukcji zapytania i kończy gdy któraś instrukcja (lub \texttt{unify}) zwróci błąd albo maszyna dojdzie do końca kodu. W pierwszym wypadku ustawiana jest flaga \texttt{fail} i zaczyna się nawrót, czyli próba przywrócenia maszyny do stanu z ostatniego punktu wyboru. Jeśli taki punkt wyboru istnieje to wykonywanie wznawia się w miejscu w kodzie wskazanym przez dany punkt. W przeciwnym wypadku wykonywanie kończy się porażką i aplikacja wyświetla wartość wyjściową $fałsz$. Jeżeli maszyna dojdzie do końca kodu, to wykonanie kończy się sukcesem i aplikacja pyta użytkownika czy kontynuować wykonywanie. Jeżeli użytkownik odpowie "tak" to maszyna zachowuje się tak, jakby właśnie trafiła na błąd i zaczyna nawrót. W przeciwnym wypadku kończy wykonywanie.\\
Po zakończeniu wykonywania aplikacja usuwa ze strefy kodu kod zapytania zostawiając tylko kod wbudowanych predykatów i programu, a następnie prosi o wpisanie następnego zapytania. Aplikacja działa w pętli aż nie zostanie manualnie zatrzymana (ctrl+c).\\

\begin{minipage}{\textwidth}
Przykładowy fragment kodu odpowiedzialny za funkcję \texttt{unify} służącą do udowadniania tożsamości dwóch poddrzew unifikowanych termów:\\
\begin{verbatim}
bool unify(Address a1, Address a2){
    std::stack<Address> pdl;
    pdl.push(a1);
    pdl.push(a2);
    while(!pdl.empty()){
        Address d1 = deref(pdl.top());
        pdl.pop();
        Address d2 = deref(pdl.top());
        pdl.pop();
        if(d1 != d2){
            if(d1.getCell().tag == "REF" || d2.getCell().tag == "REF"){
                bind(d1,d2);
            } else{
                functor f1 = get_functor(d1.getCell().getAddr().getCell().tag);
                functor f2 = get_functor(d2.getCell().getAddr().getCell().tag);
                if(f1.first == f2.first && f1.second == f2.second){
                    for(unsigned int i = 1; i <= f1.second; i++){
                        pdl.push(d1.getCell().getAddr()+i);
                        pdl.push(d2.getCell().getAddr()+i);
                    }
                } else{
                    return true;
                }
            }
        }
    }
    return false;
}
\end{verbatim}
\end{minipage}

\section{Implementacja kompilatora}

Kompilator składa się z leksera napisanego przy pomocy Flexa i parsera napisanego przy pomocy GNU Bisona. Gramatyka kompilatora jest opisana w poprzednim rozdziale. Kompilator pobiera kod Prologa jako \texttt{std::string} i zwraca listę instrukcji jako \texttt{std::string}. Lekser przetwarza cały wejściowy ciąg znaków na tokeny, usuwając przy tym białe znaki i komentarze (wszystko od znaku \% do końca linii). Tokeny reprezentują statyczne znaki lub krótkie ciągi znaków używane w kodzie Prologa nie licząc tokenów \texttt{VAR} i \texttt{STRUCT}, które odpowiednio oznaczają nazwy zmiennych i struktur.\\
Parser składając termy predykatów lub zapytania konstruuje termy w postaci drzewa i zapamiętuje je aż do momentu złożenia ich w predykat lub zapytanie, kiedy generuje instrukcje. Generowanie instrukcji jest możliwe dopiero w tym momencie, ponieważ poznane zostają wszystkie zmienne i jest możliwe wyodrębnienie zmiennych, które muszą być przechowywane w rejestrach trwałych. W przypadku programu w momencie złożenia predykatów są one grupowane i generowane są instrukcje zarządzające punktami wyboru, tam gdzie jest więcej niż jeden predykat o takiej samej nazwie. Rejestry są indeksowane od 0. Instrukcje generowane dla danego termu różnią się znacząco w zależności gdzie jest ten term. Jeśli term jest w zapytaniu lub w ciele reguły to generowane są dla niego instrukcje konstruujące reprezentacje termu w maszynie i umieszczające ją w argumentach (np. \texttt{put\_structure}, \texttt{set\_variable} i \texttt{set\_value}). Instrukcje dla struktur w takim termie są generowane od liści do korzenia. Każdy term jest konstruowany ze skonstruowanych już podtermów. Za to instrukcje generowane dla termów z faktów lub głowy reguły służą do zczytywania istniejących termów i sprawdzania ich zgodności z obecnie rozpatrywanym termem (np. \texttt{get\_structure}, \texttt{unify\_variable} i \texttt{unify\_value}). W tym wypadku instrukcje dla struktur są generowane w kolejności od korzenia do liści. Argumenty tymczasowe są przydzielane do termów w danym drzewie w kolejności zgodnej z przeszukaniem w szerz. Rejestry argumentów są przydzielane zgodnie z kolejnością argumentów, a rejestry trwałe z kojenością drugich wystąpień zmiennych w predykacie lub zapytaniu.\\

\begin{minipage}{\textwidth}
Przykładowy fragment kodu odpowiedzialny za składanie predykatów w program:\\
\begin{verbatim}
{
std::unordered_map<std::string, std::vector<unsigned int>> clauses;
for(unsigned int i=0; i<$1.ts.size(); i++){
       std::string name = $1.ts[i]->name + "/" + std::to_string($1.ts[i]->no_subterms);
       if(clauses.find(name) == clauses.end()){
             clauses[name] = std::vector<unsigned int>();
       }
       clauses[name].push_back(i);
}
int labels = 0;
std::string code;
for(std::pair<std::string, std::vector<unsigned int>> clause : clauses){
       code += clause.first + " :\n";
       if(clause.second.size() == 1){
              code += $1.codes[clause.second[0]];
       } else{
              for(unsigned int i=0; i<clause.second.size(); i++){
                     if(i == 0){
                            code += "try_me_else L" + std::to_string(labels) + "\n";
                     } else{
                            code += "L" + std::to_string(labels) + " :\n";
                            labels++;
                            if(i == clause.second.size()-1){
                                   code += "trust_me\n";
                            } else{
                                   code += "retry_me_else L" + std::to_string(labels) + "\n";
                            }
                     }
                     code += $1.codes[clause.second[i]];
              }
       }
}
ostring = code;
YYACCEPT;
}
\end{verbatim}
\end{minipage}