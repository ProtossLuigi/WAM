\chapter{Kompilator}
\thispagestyle{chapterBeginStyle}

\iffalse
W tym rozdziale należy omówić zawartość pakietu instalacyjnego oraz założenia co do środowiska, w którym realizowany system będzie instalowany. Należy przedstawić procedurę instalacji i wdrożenia systemu. Czynności instalacyjne powinny być szczegółowo rozpisane na kroki. Procedura wdrożenia powinna obejmować konfigurację platformy sprzętowej, OS (np. konfiguracje niezbędnych sterowników) oraz konfigurację wdrażanego systemu, m.in.\ tworzenia niezbędnych kont użytkowników. Procedura instalacji powinna prowadzić od stanu, w którym nie są zainstalowane żadne składniki systemu, do stanu w którym system jest gotowy do pracy i oczekuje na akcje typowego użytkownika.
\fi

W tym rozdziale opiszę działanie i sposób obsługi kompilatora.

\section{Gramatyka}

Tu umieszczę schemat gramatyki.

\section{Sposób alokacji pamięci}

Tutaj opiszę w jaki sposób kompilator decyduje o przydziale rejestrów.